\documentclass[11pt]{article}
\usepackage{fullpage}

\title{LaTeX notation template}
\author{S520}
\date{}

\begin{document}

\maketitle

\section*{Permutations and combinations}

\begin{eqnarray*}
P(n, r) &=& \frac{n!}{(n - r)!} \\
C(n, r) &=& \frac{n!}{r! (n - r)!}
\end{eqnarray*}

You can also use bracket notation for combinations:

\[
{{n}\choose{r}} = \frac{n!}{r! (n - r)!}
\]

\section*{Piecewise functions}

\[
f(x) = \left\{
\begin{array}{cl}
0.2 & 0 \le x < 3 \\
0.1 & 3 \le x < 7 \\
0 & \textrm{otherwise}
\end{array}
\right.
\]

\section*{Probabilities}

I just write $P(X \le a)$, $P(X < a)$, $P(X \ge a)$, $P(X > a)$, $P(a < X < b)$.

\section*{Expectation and variance}

Expected value of discrete $X$:
\[
E(X) \equiv EX = \sum_x x \cdot f(x).
\]
Expected value of $X^2$:
\[
E(X^2) = \sum_x x^2 \cdot f(x).
\]

\end{document}




