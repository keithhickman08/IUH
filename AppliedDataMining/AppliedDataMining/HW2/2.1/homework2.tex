%%%%%%%%%%%%%%%%%%%%%%%%%%%%%%%%%%%%%%%%%
% Programming/Coding Assignment
% LaTeX Template
%
% This template has been downloaded from:
% http://www.latextemplates.com
%
% Original author:
% Ted Pavlic (http://www.tedpavlic.com)
%
% Note:
% The \lipsum[#] commands throughout this template generate dummy text
% to fill the template out. These commands should all be removed when 
% writing assignment content.
%
% This template uses a Perl script as an example snippet of code, most other
% languages are also usable. Configure them in the "CODE INCLUSION 
% CONFIGURATION" section.
%
%%%%%%%%%%%%%%%%%%%%%%%%%%%%%%%%%%%%%%%%%

%----------------------------------------------------------------------------------------
%	PACKAGES AND OTHER DOCUMENT CONFIGURATIONS
%----------------------------------------------------------------------------------------

\documentclass{article}

\usepackage{fancyhdr} % Required for custom headers
\usepackage{lastpage} % Required to determine the last page for the footer
\usepackage{extramarks} % Required for headers and footers
\usepackage[usenames,dvipsnames]{color} % Required for custom colors
\usepackage{graphicx} % Required to insert images
\usepackage{listings} % Required for insertion of code
\usepackage{courier} % Required for the courier font
\usepackage{lipsum} % Used for inserting dummy 'Lorem ipsum' text into the template
\usepackage{url}
% Margins
\topmargin=-0.45in
\evensidemargin=0in
\oddsidemargin=0in
\textwidth=6.5in
\textheight=9.0in
\headsep=0.25in

\linespread{1.1} % Line spacing

% Set up the header and footer
\pagestyle{fancy}
\lhead{\hmwkAuthorName} % Top left header
\chead{\hmwkClass\ (\hmwkClassInstructor\ \hmwkClassTime): \hmwkTitle} % Top center head
\rhead{\firstxmark} % Top right header
\lfoot{\lastxmark} % Bottom left footer
\cfoot{} % Bottom center footer
\rfoot{Page\ \thepage\ of\ \protect\pageref{LastPage}} % Bottom right footer
\renewcommand\headrulewidth{0.4pt} % Size of the header rule
\renewcommand\footrulewidth{0.4pt} % Size of the footer rule

\setlength\parindent{0pt} % Removes all indentation from paragraphs

%----------------------------------------------------------------------------------------
%	CODE INCLUSION CONFIGURATION
%----------------------------------------------------------------------------------------

\definecolor{MyDarkGreen}{rgb}{0.0,0.4,0.0} % This is the color used for comments
\lstloadlanguages{R} % Load Perl syntax for listings, for a list of other languages supported see: ftp://ftp.tex.ac.uk/tex-archive/macros/latex/contrib/listings/listings.pdf
\lstset{language=R, % Use Perl in this example
        frame=single, % Single frame around code
        basicstyle=\small\ttfamily, % Use small true type font
        keywordstyle=[1]\color{Blue}\bf, % Perl functions bold and blue
        keywordstyle=[2]\color{Purple}, % Perl function arguments purple
        keywordstyle=[3]\color{Blue}\underbar, % Custom functions underlined and blue
        identifierstyle=, % Nothing special about identifiers                                         
        commentstyle=\usefont{T1}{pcr}{m}{sl}\color{MyDarkGreen}\small, % Comments small dark green courier font
        stringstyle=\color{Purple}, % Strings are purple
        showstringspaces=false, % Don't put marks in string spaces
        tabsize=5, % 5 spaces per tab
        %
        % Put standard Perl functions not included in the default language here
        morekeywords={},
        %
        % Put Perl function parameters here
        morekeywords=[2]{on, off, interp},
        %
        % Put user defined functions here
        morekeywords=[3]{test},
       	%
        morecomment=[l][\color{Blue}]{...}, % Line continuation (...) like blue comment
        numbers=left, % Line numbers on left
        firstnumber=1, % Line numbers start with line 1
        numberstyle=\tiny\color{Blue}, % Line numbers are blue and small
        stepnumber=5 % Line numbers go in steps of 5
}

% Creates a new command to include a perl script, the first parameter is the filename of the script (without .pl), the second parameter is the caption
\newcommand{\rscript}[2]{
\begin{itemize}
\item[]\lstinputlisting[caption=#2,label=#1]{#1.R}
\end{itemize}
}

%----------------------------------------------------------------------------------------
%	DOCUMENT STRUCTURE COMMANDS
%	Skip this unless you know what you're doing
%----------------------------------------------------------------------------------------

% Header and footer for when a page split occurs within a problem environment
\newcommand{\enterProblemHeader}[1]{
\nobreak\extramarks{#1}{#1 continued on next page\ldots}\nobreak
\nobreak\extramarks{#1 (continued)}{#1 continued on next page\ldots}\nobreak
}

% Header and footer for when a page split occurs between problem environments
\newcommand{\exitProblemHeader}[1]{
\nobreak\extramarks{#1 (continued)}{#1 continued on next page\ldots}\nobreak
\nobreak\extramarks{#1}{}\nobreak
}

\setcounter{secnumdepth}{0} % Removes default section numbers
\newcounter{homeworkProblemCounter} % Creates a counter to keep track of the number of problems

\newcommand{\homeworkProblemName}{}
\newenvironment{homeworkProblem}[1][Problem \arabic{homeworkProblemCounter}]{ % Makes a new environment called homeworkProblem which takes 1 argument (custom name) but the default is "Problem #"
\stepcounter{homeworkProblemCounter} % Increase counter for number of problems
\renewcommand{\homeworkProblemName}{#1} % Assign \homeworkProblemName the name of the problem
\section{\homeworkProblemName} % Make a section in the document with the custom problem count
\enterProblemHeader{\homeworkProblemName} % Header and footer within the environment
}{
\exitProblemHeader{\homeworkProblemName} % Header and footer after the environment
}

\newcommand{\problemAnswer}[1]{ % Defines the problem answer command with the content as the only argument
\noindent\framebox[\columnwidth][c]{\begin{minipage}{0.98\columnwidth}#1\end{minipage}} % Makes the box around the problem answer and puts the content inside
}

\newcommand{\homeworkSectionName}{}
\newenvironment{homeworkSection}[1]{ % New environment for sections within homework problems, takes 1 argument - the name of the section
\renewcommand{\homeworkSectionName}{#1} % Assign \homeworkSectionName to the name of the section from the environment argument
\subsection{\homeworkSectionName} % Make a subsection with the custom name of the subsection
\enterProblemHeader{\homeworkProblemName\ [\homeworkSectionName]} % Header and footer within the environment
}{
\enterProblemHeader{\homeworkProblemName} % Header and footer after the environment
}

%----------------------------------------------------------------------------------------
%	NAME AND CLASS SECTION
%----------------------------------------------------------------------------------------

\newcommand{\hmwkTitle}{Homework\ \#2} % Assignment title
\newcommand{\hmwkDueDate}{9/6/2017} % Due date
\newcommand{\hmwkClass}{Applied Datamining} % Course/class
\newcommand{\hmwkClassTime}{Online} % Class/lecture time
\newcommand{\hmwkClassInstructor}{Instructor: Hasan Kurban} % Teacher/lecturer
\newcommand{\hmwkAuthorName}{Keith Hickman} % Your name

%----------------------------------------------------------------------------------------
%	TITLE PAGE
%----------------------------------------------------------------------------------------

\title{
\vspace{2in}
\textmd{\textbf{\hmwkClass:\ \hmwkTitle}}\\
\normalsize\vspace{0.1in}\small{Due\ on\ \hmwkDueDate}\\
\vspace{0.1in}\large{\textit{\hmwkClassInstructor\ }}
\vspace{3in}
}

\author{\textbf{\hmwkAuthorName}}
\date{\today} % Insert date here if you want it to appear below your name

%----------------------------------------------------------------------------------------

\begin{document}

\maketitle

%----------------------------------------------------------------------------------------
%	TABLE OF CONTENTS
%----------------------------------------------------------------------------------------

%\setcounter{tocdepth}{1} % Uncomment this line if you don't want subsections listed in the ToC

\newpage
\tableofcontents
\newpage

%----------------------------------------------------------------------------------------
%	PROBLEM 1
%----------------------------------------------------------------------------------------

% To have just one problem per page, simply put a \clearpage after each problem

\begin{homeworkProblem}
For the following data, give the best taxonomic type (interval, ratio, nominal, ordinal):
\begin{enumerate}
\item A section of highway on a map.
	- Interval
\item The value of a stock.
	- Ratio
\item The weight of a person.
	- Ratio, as we can be infinitely precise about a person's weight. 
\item Marital status.
	- Ordinal.  (One would have to be single before married, married before divorced. Thus there is some implicit ordering among the variable.
\item Visiting United Airlines (\url{https://www.united.com}) the seating is: Economony, Economy plus, and United Business.
	- Ordinal.
\end{enumerate}
\end{homeworkProblem}

\begin{homeworkProblem}
You are datamining with a column that has physical addresses in some city with the same zipcode.  For example,
\begin{verbatim}

55 WEST CIR
2131 South Creek Road
Apt. #1 Fountain Park
1114 Rosewood Cir
1114 Rosewood Ct.
1114 Rosewood Drive

\end{verbatim}

What structure would you create to mine these?  What questions do you think you should be able to answer?
A:  It depends on the problem set.  There are several types of problems that might use address data, including crime statistics, general mapping applications 
real estate sales, or school districting.  I would create different structures for each problem (crime -  time and geolocation; school - assigned schools, ratings; real estate sales - home characteristics and other homes sold information).

\end{homeworkProblem}


\begin{homeworkProblem}
The Wisconsin Breast Cancer data set is very famous. Here is the URL \url{https://archive.ics.uci.edu/ml/datasets/breast+cancer+wisconsin+(original)}.  In the Data Folder are multiple files.  Here is the beginning of an R session that allows us to read this data from the web into our local R session:
\begin{verbatim}
> install.packages("data.table")
> library(data.table)
> install.packages("curl")
> mydata <- fread("https://archive.ics.uci.edu/ml/machine-learning-databases/
                   breast-cancer-wisconsin/breast-cancer-wisconsin.data")
> head(mydata)
        V1 V2 V3 V4 V5 V6 V7 V8 V9 V10 V11
1: 1000025  5  1  1  1  2  1  3  1   1   2
2: 1002945  5  4  4  5  7 10  3  2   1   2
3: 1015425  3  1  1  1  2  2  3  1   1   2
4: 1016277  6  8  8  1  3  4  3  7   1   2
5: 1017023  4  1  1  3  2  1  3  1   1   2
6: 1017122  8 10 10  8  7 10  9  7   1   4
> 
\end{verbatim}
\subsection{Discussion of Data}
Briefly describe this data set--what is its purpose?  How should it be used? What are the kinds of data it's using?
A: The data is used in various applications, including better prediction and diagnostic models for certain types of breast cancer.  
The data observations are individual patient visits/dianostic tests, and the dimensions are the values held for different types of diagnostic tests. 
This data has been used in various supervised/unsupervised machine learning tasks.  
Data are mostly continuous variables, likely ordinal as the higher or lower values could better explain variance or outcomes. 

\subsection{R Code}
 Using R, show code that answers the following questions:
\begin{enumerate} 
\item How many entries are in the data set? Answer here $\ldots$
\rscript{sb}{Sample R Script With Highlighting}

\begin{verbatim}
699 observations
\end{verbatim}


\item How many unknown or missing data are in the data set? Answer here $\ldots$
\rscript{sb2}{Sample R Script With Highlighting}
\begin{verbatim}
There are no missing data.
\end{verbatim}
\item How many malignant and benign identifiers are there? Answere here $\ldots$
\rscript{sb3}{Sample R Script With Highlighting}
\begin{verbatim}
There are two identifiers of malignancy in the feature labelled V11.  
4 = Malignant and 2 = Benign. 
There are 241 Malignant cases and 458 benign cases. 
\end{verbatim}
\item Make a histogram of each attribute and discuss the distribution of values \textit{e.g.}, are uniform, skewed, normal.  Place images of these histograms into the document.   Show the R code that you used below and discussion below that. 
\rscript{sb4}{Sample R Script With Highlighting}

\subsection{Discussion of Attributes}
\item Almost all of the data are right-skewed toward lower values of each variable.  I would begin standardizing the features, to control for outliers and attempt to normalize the distribution, and then try square root, log, or reciprocal transformations.  

\subsection{Histograms}

\end{enumerate} 
\subsection{Discussion of simply removing tuples}
Quantify the effect of simply removing the tuples with unknown or missing values.  What is the cost in human capital?

In cancer research, removing values with unknown or missing data can be detrimental to the overall outcomes.  When creating a model that is designed to diagnose a potentially fatal disease, 
the models must be extremely accurate.  The difference between a 99.5 and 99.99 percent accuracy could cost thousands of dollars in unncessary tests, or worse, a missed positive diagnosis. 

\end{homeworkProblem}

%----------------------------------------------------------------------------------------

\end{document}